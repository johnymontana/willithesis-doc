% !TeX encoding=utf8
% !TeX spellcheck = en-US

\chapter{Methods}
Here we describe the data used and detail the implementation. We first examine in detail two methods for link prediction: collaborative filtering using the Jaccard similarity metric and the Triadic Closeness method. We then show how these two methods can be used together in an ensemble predictor using adaptive weights. The data used in this experiment is examined. Finally, implementation is discussed. 

\section{Algorithms}


\subsection{Sample Graph}
Consider the network shown in figure \ref{thesis_sample_network}. This is a sample network which was randomly generated and does not represent any real world observed data, however we shall refer to this network to demonstrate the techniques used in this project.



\begin{figure}[H]
  \centering
  \includegraphics[width=0.75\textwidth]{images/thesis_sample_network_multimodal.png}
  \caption[Sample network]{Sample social network of hypothetical }
  \label{thesis_sample_network}
\end{figure}


\begin{table}[t]
\caption{Descriptive statistics for sample network}
\label{sample_network_stats}
\vskip 0.15in
\begin{center}
\begin{small}
\begin{sc}
\begin{tabular}{rc}
\hline
Metric & value\\
\hline
Num nodes & 17\\
Num edges & 28\\
Avg degree & 1.67\\
OTHERS & ??\\
\hline
\end{tabular}
\end{sc}
\end{small}
\end{center}
\vskip -0.1in
\end{table}

We will generate recommendations for a user in the sample network using each of the three methods: 1) collaborative filtering, 2) triadic closeness, and 3) a new approach combining triadic closeness and multi-modal relationship types that has been presented here for the first time.

As mentioned previously, generating recommendations can be thought of as link prediction: we are identifying links that do not appear in the network and predicting links that will either 1) appear in the future or 2) in the case of a partially observed network, links that exist but are not observed.

For the purposes of the next three sections we will consider link prediction for user J. We proceed through each algorithm manually, ignoring some implementation details for now that will explored in depth in the proceeding section.

\subsection{Collaborative Filtering}
TODO: REWRITE THIS PIECE TO MAKE USE OF REPOS / STARS

Collaborative filtering is a method of generating recommendations based on the homophily principle: users who are similar are likely to be interested in similar items. It is implemented by finding similar users, allowing each similar user to "vote" for recommendations and suggested items with the highest score. This is similar to the kNN algorithm used for classification.\cite{cf}

Collaborative filtering is usually used in the context of user-item recommendation, so here we simply treat other users as items (as necessary).


\subsubsection{Similarity Metrics}
The Jaccard index is used to identify similar users. For two users, $a$ and $b$, let $A$ and $B$ denote the sets of all users being followed by $a$ and $b$, respectively. The Jaccard index is therefore as defined in Equation \ref{jaccard}.
\begin{equation}
\label{jaccard}
J(A,B) = \frac{|A \cap B|}{|A \cup B|}
\end{equation}

In this context, Jaccard is defined as the intersection of the users followed by $a$ and $b$ divided by the union of the users followed by $a$ and $b$. This results in a number between 0 and 1, indicating the strength of similarity between users $a$ and $b$.
 
%\begin{algorithm}[tb]
%
%  \caption{Collaborative Filtering Recommendation}
%   \label{alg:cf}
%\begin{algorithmic}
%   \STATE {\bfseries Input:} Graph $G(V,E)$, vertex $v$, int k
%   
%   \FOR{$i=1$ {\bfseries to} $m-1$}
%   \IF{$x_i > x_{i+1}$} 
%   \STATE Swap $x_i$ and $x_{i+1}$
%   \STATE $noChange = false$
%   \ENDIF
%   \ENDFOR
%  
%\end{algorithmic}
%\end{algorithm}

\begin{algorithm}
\caption{Collaborative filtering algorithm}\label{algo1}
\begin{algorithmic}[1]
\State $input: G(U, E), x, N$ \Comment{explain inputs here}
\State $usersSample \gets getRandomUsers(U,x)$
\State $results \gets \{\}$
\For {$each user in usersSample$}
	\State $validationEdge \gets getRandomEdge(G, user)$
	\State $removeEdge(validationEdge,G)$
	\State fofs $\gets getFOFS(user, G)$
	\State fofRanks $ \gets \{\}$
	\For{each fof in fofs}
		\State j $\gets jaccard(user, fof)$
		fofRanks $\gets results + \{j : fof\}$
	\EndFor
	\State knn $\gets topk(fofRanks, k)$
	\State aggregated $\gets \{\}$
	\For{each kn in knn}
		\State possible $\gets getFollows(kn)$
		\For{each p in possible}
			\If{p in aggregated.keys} $aggregated[p] += 1$ 
				%\State 
			\Else 
				\State $aggregated[p] = 1$
			\EndIf
		\EndFor
	\EndFor
	
	\State $predictions \gets topXSortedByTC(pred, N)$
	\State $hit \gets is validationEdge in predictions?$
	\State $addEdge(validationEdge, G)$
	\State $results \gets results + \{hit, pred, u, v, validation_edge\}$
\EndFor
\State \Return results
\end{algorithmic}
\end{algorithm}

To generate recommendations for user $J$, we first must identify all friend-of-friend nodes, that is nodes that share a neighbor in common with $J$. That gives us the set $\{G, L\}$. Our possible recommendations are now reduced to G, L. We will now compute the Jaccard similarity metric for the pairs $(J, G)$ and $(J, L)$:

\begin{equation}
\label{jaccard}
J(J,G) = \frac{|J \cap G|}{|J \cup G|}
\end{equation}

\begin{equation}
\label{jaccard}
J(J,G) = \frac{|\{F\}|}{|\{P, F, K\}|}
\end{equation}

\begin{equation}
\label{jaccard}
J(J,G) = \frac{1}{3}
\end{equation}

and for J(J,L):

TODO: adjust network so Jaccard calculation is different

\begin{equation}
\label{jaccard}
J(J,L) = \frac{|J \cap L|}{|J \cup L|}
\end{equation}

\begin{equation}
\label{jaccard}
J(J,G) = \frac{|\{P\}|}{|\{F, P, Q\}|}
\end{equation}

\begin{equation}
\label{jaccard}
J(J,G) = \frac{1}{3}
\end{equation}

We now take the top $k$ nodes that have the highest Jaccard score and allow each to vote for new outgoing links to form from $J$. Here we will select $L$ and recommend any outgoing links from $L$ as destination nodes for predicted links emminating from $J$. In this case we predict the link $J \rightarrow Q$.

As you can see, the collaborative filtering link prediction process for a given user $x$ involves finding other users most similar to user $x$, then finding items those similar users are most interested in. In this sense collaborative filtering can be thought of as very similar to k-nearest neighbors, where the distance calculation is based on some similarity metric.




\subsection{Triadic Closeness}
Graph theory proposes the concept of triadic closure, the hypothesis that the creation of an edge between u and v is related to the degree of overlapping neighbors in u and v's respective networks. (CITATION NEEDED and expand on this - see one of the network analysis books I have) The concept of Triadic Closeness is an application of the theory of triadic closure, specifically taking into account the directed nature of social networks. For a given fully observed network, Triadic Closeness can be thought of as the ratio of the number of closed triads to the number of potentially closed triads. (CITATION NEEDED Schall 2013). A triad consists of three nodes $u, z, v$ where edges (ignoring direction) $u,z$ and $z,v$ exist. Edges between $u$ and $v$ may exist, however the concept of triadic closure posits that an implicit connection exists between $u$ and $v$.

\begin{figure}[H]
  \centering
  \includegraphics[width=0.75\textwidth]{images/thesis_triad_example.png}
  \caption[closed triad patterns]{A triad is considered to be closed if an edge exists between u and v.}
  \label{thesis_closed_triads}
\end{figure}


In a directed network there are 27 distinct configurations, or patterns that a triad can take on. See (CITATION TABLE of triad patterns). Table XXXXX (CITE ME) shows triad patterns that are open, that is no connection exists between nodes $u$ and $v$. The pattern identifications $(T01, T02...)$ are taken from (CITE PROPERLY) Schall 2013. Any open triad can be closed in one of three possible ways: $u \leftarrow v$, $u \rightarrow v$, or $u \leftrightarrow v$.

INSERT IMAGE OF 3 CLOSED TRIAD PATTERNS 
\begin{figure}[H]
  \centering
  \includegraphics[width=0.75\textwidth]{images/thesis_triad_patterns.png}
  \caption[triad patterns]{All possible open triad patterns are shown. A triad is said to be closed is a connection is established between u and v.}
  \label{thesis_triad_patterns}
\end{figure}

\begin{figure}[H]
  \centering
  \includegraphics[width=0.75\textwidth]{images/thesis_closed_triads.png}
  \caption[closed triad patterns]{A triad is considered to be closed if an edge exists between u and v.}
  \label{thesis_closed_triads}
\end{figure}

Figure \ref{thesis_closed_triads} shows the three distinct ways in which an open triad can be closed. Either the creation of an edge $u \rightarrow u$, the creation of an edge $u \leftarrow v$, or the creation of two edges $u \rightarrow v$ and $u \leftarrow v$. In terms of the triad pattern id, the first digit indicates if the triad is open (0), closed with an edge $u \rightarrow v$ (1), closed with an edge $u \leftarrow v$ (2), or closed with two edges $u \rightarrow v$ and $u \leftarrow v$ (3). With this nomenclature we are now able to represent each possible triad pattern with a two digit identifier. 

\begin{table}[t]
\caption{Open triad pattern frequency in the sample network shown in Figure CITATION NEEDED}
\label{sample_network_freq}
\vskip 0.15in
\begin{center}
\begin{small}
\begin{sc}
\begin{tabular}{rrccr}
\hline
ID & Pattern & Count \\
\hline
T06 & $u \rightarrow z \leftarrow v $ & 20 \\
T04 &$ u \rightarrow z \rightarrow v$ & 14 \\
T08 & $u \leftarrow z \leftarrow v$ & 14 \\
T02 & $u \leftrightarrow z \rightarrow v$ & 8 \\
T07 & $u \leftarrow z \leftrightarrow v$ & 8 \\
T09 & $u \leftarrow z \rightarrow v$ & 8 \\
T03 & $u \rightarrow z \leftrightarrow v$ & 3 \\
T05 & $u \leftrightarrow z \leftarrow v$ & 3 \\

\hline
\end{tabular}
\end{sc}
\end{small}
\end{center}
\vskip -0.1in
\end{table}

\begin{table}[t]
\caption{Closed triad pattern frequency in the sample network shown in Figure CITATION NEEDED}
\label{sample_network_freq}
\vskip 0.15in
\begin{center}
\begin{small}
\begin{sc}
\begin{tabular}{rrccr}
\hline
ID & Pattern & Count \\
\hline
T18 & $u \leftarrow z \leftarrow v \leftarrow u$ & 3 \\
T14 & $u \rightarrow z \rightarrow v \leftarrow u$ & 2 \\
T16 & $u \rightarrow z \leftarrow v \leftarrow u$ & 2 \\
T19 & $u \leftarrow z \rightarrow v \leftarrow u$ & 2 \\
T15 & $u \leftrightarrow z \leftarrow v \leftarrow u$ & 1 \\
T17 & $u \leftarrow z \leftrightarrow v \leftarrow u$ & 1 \\
T24 & $u \rightarrow z \rightarrow v \rightarrow u$ & 3 \\
T26 & $u \rightarrow z \leftarrow v \rightarrow u$ & 2 \\
T28 & $u \leftarrow z \leftarrow v \rightarrow u$ & 2 \\
T29 & $u \leftarrow z \rightarrow v \rightarrow u$ & 2 \\
T22 & $u \leftrightarrow z \rightarrow v \rightarrow u$ & 1 \\
T23 & $u \rightarrow z \leftrightarrow v \rightarrow u$ & 1 \\
T34 & $u \rightarrow z \rightarrow v \leftrightarrow u$ & 1 \\
T38 & $u \leftarrow z \leftarrow v \leftrightarrow u$ & 1 \\
\hline
\end{tabular}
\end{sc}
\end{small}
\end{center}
\vskip -0.1in
\end{table}

\begin{equation}
TC_{uv} = \sum_{z\in \Gamma(u)\cap \Gamma(v)} w^{P}(u, v, z) \times w(z)  
\end{equation}

\begin{equation}
w^{P}(u,v,z) = \frac{F(T(u,v,z) + 10) + F(T(u,v,z) + 30)}{F(T(u,v,z))}
\end{equation}

\begin{equation}
TC_{uv} = \sum_{z\in \Gamma(u)\cap \Gamma(v)} w^{P}(u,v,z) \times \frac{1}{k_z}
\end{equation}

Having collected these triad pattern frequencies, we can now generate recommendations as we did in section (REFERENCE SECTION COLLABORATIVE FILTERING) for user $J$.

The first step is identify all open triads of the form u,z,v where J is u and no link between u,z exists. Those triads are:

J,F,G
J,I,N
J,P,Q

(INSERT GRAPH HERE OF THE THREE TRIADS AND THEIR PATTERNS)

Thus the three possible recommendations that we might generate are $J \rightarrow G$, $J \rightarrow N$, and $J \rightarrow Q$. To determine the rank of the predictions, we must calculate the triadic closeness metric for the pairs $(J,G)$, $(J,N)$ and $(J,A)$. 

\begin{equation}
TC_{JG} = \sum_{z\in \Gamma(J)\cap \Gamma(G)} w^{P}(J, G, z) \times w(z)  
\end{equation}

\begin{equation}
TC_{JG} = w^{P}(J, G, F) \times w(F)  
\end{equation}

\begin{equation}
w^{P}(J,G,F) = \frac{F(T(J,F,G) + 10) + F(T(J,F,G) + 30)}{F(T(J,F,G))}
\end{equation}

We can see that T(J,F,G) = T03, so we have:

\begin{equation}
w^{P}(J,G,F) = \frac{F(T03 + 10) + F(T03 + 30)}{F(T03)}
\end{equation}

\begin{equation}
w^{P}(J,G,F) = \frac{0 + 0}{3}
\end{equation}

\begin{equation}
w^{P}(J,G,F) = 0
\end{equation}

\begin{equation}
TC_{JG} = 0
\end{equation}

Next for J,N:

\begin{equation}
TC_{JG} = \sum_{z\in \Gamma(J)\cap \Gamma(G)} w^{P}(J, N, z) \times w(z)  
\end{equation}

\begin{equation}
TC_{JG} = w^{P}(J, I, N) \times w(I)  
\end{equation}

\begin{equation}
w^{P}(J,G,F) = \frac{F(T(J,I,N) + 10) + F(T(J,I,N) + 30)}{F(T(J,I,N))}
\end{equation}

We can see that T(J,I,N) = T08, so we have:

\begin{equation}
w^{P}(J,G,F) = \frac{F(T08 + 10) + F(T08 + 30)}{F(T08)}
\end{equation}

Using table (TRIAD FREQ PATTERN REF TABLE)

\begin{equation}
w^{P}(J,G,F) = \frac{3 + 1}{14}
\end{equation}

\begin{equation}
w^{P}(J,G,F) = 0.286
\end{equation}

\begin{equation}
w(I)=1/3
\end{equation}

\begin{equation}
TC_{JN} = 0.286 \times 1/3 = 0.095
\end{equation}


and finally, calculate TC(J,Q):


\begin{equation}
TC_{JQ} = \sum_{z\in \Gamma(J)\cap \Gamma(Q)} w^{P}(J, Q, z) \times w(z)  
\end{equation}

\begin{equation}
TC_{JG} = w^{P}(J, P, A) \times w(Q)  
\end{equation}

\begin{equation}
w^{P}(J,P,Q) = \frac{F(T(J,P,Q) + 10) + F(T(J,P,Q) + 30)}{F(T(J,P,Q))}
\end{equation}

We can see that T(J,P,Q) = T04, so we have:

\begin{equation}
w^{P}(J,P,Q) = \frac{F(T04 + 10) + F(T04 + 30)}{F(T04)}
\end{equation}

\begin{equation}
w^{P}(J,P,Q) = \frac{2 + 0}{14}
\end{equation}

\begin{equation}
w^{P}(J,P,Q) = 0.143
\end{equation}

\begin{equation}
w(P)=1/3
\end{equation}

\begin{equation}
TC_{JQ} = 0.143 \times 1/3
\end{equation}

\begin{equation}
TC_{JQ} = .048
\end{equation}

We can then sort our recommendations by TC and the most likely edge we will recommend is $J \rightarrow Q$.

\subsection{An adaptive ensemble method}
Linear combination of similarity metric (Jaccard) and network based method (triadic closeness).
Why might this be beneficial?

\begin{figure}[H]
  \centering
  \includegraphics[width=0.75\textwidth]{images/thesis_sample_network_multimodal.png}
  \caption[Adding multi modal edges to sample network]{We now }
  \label{thesis_triad_patterns}
\end{figure}

\begin{equation}
new_shit_{uv} = \sum_{z\in \Gamma(u)\cap \Gamma(v)} w^{P}(u, v, z) \times w(z)  
\end{equation}



\section{Implementation}


\subsection{Graph Data Model}
The labeled property graph

\subsection{Graph Traversal Pattern}
What is the graph traversal pattern? The graph traversal pattern allows us to design algorithms where the answer to our question is a traversal through the graph. 

\subsection{Graph Database And Querying Traversals}
Why using a graph database makes sense.
Performance benefits of using graph database.
Compare to how this would be modeled in a RDBMS or a document DB.
\subsection{}
\begin{figure}[ht]
\vskip 0.2in
\begin{center}
\centerline{\includegraphics[width=\columnwidth]{images/jaccard.png}}
\caption{Example of Neo4j Cypher query language used to calculate Jaccard metric for pairs of users.}
\label{icml-historical}
\end{center}
\vskip -0.2in
\end{figure} 
The system is developed using Java and makes use of the Neo4j graph database. The Cypher query language is used to query the graph database and for defining graph traversals for processing. The program is configurable and implements k-fold cross validation.

\subsection{Triad detection}

%\label{sec:example:code}
%
%\ifcsdef{lstStyleLaTeX}{%
%  \lstinputlisting[style=lstStyleLaTeX,%nolol=true,%
%     caption={LaTeX Typesetting By Example}, label=lstLaTeXExample]  
%  {content/demo/latextutorial.tex}
%}{}


Triad detection involves scanning the entire graph to identify triads in the network, both open and closed. This process is 
\subsection{Steps}
The following describes implementation of the system at a high level. For each validation fold:

\begin{itemize}
\item Select $p$ users at random
\item For each user $u$ in $p$:
	\begin{itemize}
	\item Remove one $FOLLOWS$ edge $f$ for this user at random
	\item Find $k$ nearest neighbors using Jaccard index (ensure removed edge is not used for this calculation)
	\item Calculate $n$ followed users with greatest overlap (the most commonly followed users among the $k$ nearest neighbors)
	\item Return set of $n$ users, these are the recommended users
	\item Is $f$ in $n$? If yes, this run is counted as a valid prediction.
	\end{itemize}
\item Report summary accuracy metrics for this fold
\end{itemize}


%caption{Collaborative Filtering Recommendation}
%\label{alg:cf}

\begin{algorithm}
\caption{Link prediction algorithm}\label{algo_tc}
\begin{algorithmic}[1]
\State $input: G(U, E), x, N$ \Comment{explain inputs here}
\State $usersSample \gets getRandomUsers(U,x)$
\State $results \gets \{\}$
\For {$each user in usersSample$}
	\State $validationEdge \gets getRandomEdge(G, user)$
	\State $removeEdge(validationEdge,G)$
	\State $triads \gets getTriads(user, G)$
	\State $pred \gets \{\}$
	\For {$u, v in triads$}
		\State $tc \gets calcTC(u,v,G)$
		\State $pred \gets pred + \{tc, u, v\}$
	\EndFor
	\State $predictions \gets topXSortedByTC(pred, N)$
	\State $hit \gets is validationEdge in predictions?$
	\State $addEdge(validationEdge, G)$
	\State $results \gets results + \{hit, pred, u, v, validation_edge\}$
\EndFor
\State \Return results
\end{algorithmic}
\end{algorithm}

\section{Data}

\begin{figure}[H]
  \centering
  \includegraphics[width=0.75\textwidth]{images/githubdatamodel.png}
  \caption[GitHub graph data model]{GitHub data model as a labeled property graph.}
  \label{fig:figures:1}
\end{figure}


Data was collected from GitHub Archive\cite{githubarchive}, a service that maintains an archive of all public events emitted by the GitHub API\cite{github:Online}. These include events such as creation of new repositories, pushes to repositories, repository stars, and user follows. Data was collected for the time period April 1st, 2013 - April 1st, 2014. For the purposes of this project, only \textit{FOLLOWS} events were considered. This resulted in a total of 539893 users and 919489 follows. In terms of a graph, that equates to 539893 nodes and 919489 vertices between nodes.

\begin{figure}[ht]
\vskip 0.2in
\begin{center}
\centerline{\includegraphics[width=0.3\columnwidth]{images/user_user.png}}
\caption[User-User data model]{The GitHub follow graph is a simple graph with User nodes and Follows edges.}
\label{fig:figures:4}
\end{center}
\vskip -0.2in
\end{figure} 

A graph data model is used to represent this data as the data is highly connected: it is describing entities (users and repositories) and their interactions (stars, follows, pushes, etc). Figure \ref{screenshot-data} shows an example of a subgraph of user and repository nodes and the interactions among those entities, modeled as a graph.

\subsection{Github Archive}
\label{sec:example:code}


\subsubsection{FollowEvent}
\begin{lstlisting}[style=lstStyleCpp, caption={LaTeX Typesetting By Example}, label=lstLaTeXExample]
{
  "created_at": "2013-07-11T15:03:05-07:00",
  "payload": {
    "target": {
      "id": 4602587,
      "login": "smarquez1",
      "followers": 1,
      "repos": 1,
      "gravatar_id": "42eb6556201588fa7641bf2f0bf615e6"
    }
  },
  "public": true,
  "type": "FollowEvent",
  "url": "https://github.com/smarquez1",
  "actor": "matiasalvarez87",
  "actor_attributes": {
    "login": "matiasalvarez87",
    "type": "User",
    "gravatar_id": "0ee1a5bec013545c91ad05c451fb9715",
    "name": "Matias Alvarez Duran",
    "company": "NaN Labs",
    "blog": "http://ar.linkedin.com/pub/matias-emiliano-alvarez-duran/17/39b/a96",
    "location": "Argentina",
    "email": "matiasalvarez87@gmail.com"
  }
}
\end{lstlisting}

\subsubsection{WatchEvent (Stars)}
\begin{lstlisting}[style=lstStyleCpp, caption={peep this JSON object dawg},
label=watchEventListing]
{
  "created_at": "2013-07-11T15:01:56-07:00",
  "payload": {
    "action": "started"
  },
  "public": true,
  "type": "WatchEvent",
  "url": "https://github.com/CamDavidsonPilon/Probabilistic-Programming-and-Bayesian-Methods-for-Hackers",
  "actor": "cebe",
  "actor_attributes": {
    "login": "cebe",
    "type": "User",
    "gravatar_id": "2ebfe57beabd0b9f8eb9ded1237a275d",
    "name": "Carsten Brandt",
    "company": "cebe.cc",
    "blog": "http://cebe.cc/",
    "location": "Berlin, Germany",
    "email": "mail@cebe.cc"
  },
  "repository": {
    "id": 7607075,
    "name": "Probabilistic-Programming-and-Bayesian-Methods-for-Hackers",
    "url": "https://github.com/CamDavidsonPilon/Probabilistic-Programming-and-Bayesian-Methods-for-Hackers",
    "description": "aka \"Bayesian Methods for Hackers\": An introduction to Bayesian methods + probabilistic programming with a computation/understanding-first, mathematics-second point of view. All in pure Python ;)  ",
    "homepage": "http://camdavidsonpilon.github.io/Probabilistic-Programming-and-Bayesian-Methods-for-Hackers",
    "watchers": 3353,
    "stargazers": 3353,
    "forks": 444,
    "fork": false,
    "size": 1264,
    "owner": "CamDavidsonPilon",
    "private": false,
    "open_issues": 21,
    "has_issues": true,
    "has_downloads": true,
    "has_wiki": true,
    "language": "Python",
    "created_at": "2013-01-14T07:46:28-08:00",
    "pushed_at": "2013-07-04T17:08:47-07:00",
    "master_branch": "master"
  }
}
\end{lstlisting}

\subsection{Data munging}
The data from GitHubArchive is available in streaming JSON format and includes all public events generated from the GitHub event API \cite{github:Online} for a given time period. The one year of data collected for this project resulted in several thousand JSON files, each several megabytes in size resulting in a raw dataset of several hundred gigabytes. As this project is only concerned with the user-user {\textit FOLLOWS} relationship, the data is parsed using a Python script to filter for only those {\textit FOLLOWS} events. For further efficiency, the data is reduced to an anonymized edgelist format:
\begin{verbatim}
    1	2
    3	4
    5	6
    7	8
    9	10
    11	12
    11	13
    14	15
    16	17
    18	19
    ...
\end{verbatim}
This allows for the dataset to be anonymized using only arbitrary user ids and stored as compactly as possible. In the above example (and in the context of a directed graph) the first column corresponds to the source node and the second column the destination node. So the sample above represents: user 1 follows user 2, user 3 follows user 4, ...

This edgelist file is then used to populate a Neo4j graph database instance, using only the minimal information necessary to implement our link prediction system. 

\begin{table}[t]
\caption{Dataset descriptive statistics}
\label{results}
\vskip 0.15in
\begin{center}
\begin{small}
\begin{sc}
\begin{tabular}{rrccr}
\hline
Nodes (users) & 539893  \\
Edges (:FOLLOWS relationships) & 919489 \\
Mean degree & 3.4 \\
Standard deviation of degree & 30.3 \\
\hline
\end{tabular}
\end{sc}
\end{small}
\end{center}
\vskip -0.1in
\end{table}


\begin{figure}[ht]
\vskip 0.2in
\begin{center}
\centerline{\includegraphics[width=\columnwidth]{images/neo_screenshot.png}}
\caption{GitHub data model as a property graph. Screenshot from Neo4j graph database interface.}
\label{screenshot-data}
\end{center}
\vskip -0.2in
\end{figure} 

\subsection{Data Analysis}

\begin{table}[ht]
\centering
\small\renewcommand{\arraystretch}{1.4}  
\rowcolors{1}{tablerowcolor}{tablebodycolor}
%
\captionabove[PageRank for User-User follows]{Here we see the most "central" Users per their PageRank rankings. This is based on the graph created by user-user follows edges.}
\label{follow_pagerank_table}
%
\begin{tabularx}{0.4\textwidth}{lXXX}
\hline
\rowcolor{tableheadcolor}
User & PageRank \\
\hline
funkenstein& 413.14 \\
mojombo & 300.01 \\
torvalds & 248.21 \\
rippleFoundation & 220.29 \\
visionmedia & 140.52 \\
paulirish & 129.59 \\
BYVoid & 114.29 \\
schacon & 112.17 \\
JakeWharton & 110.55 \\
defunkt & 106.86 \\
mattt & 99.38 \\
worrydream & 87.33 \\
hakimel & 83.05 \\
pjhyett & 80.89 \\
addyosmani & 80.59 \\
mbostock & 75.63 \\
mdo & 70.40 \\
LeaVerou & 66.92 \\
tekkub & 62.24 \\
nf & 60.93 \\

\hline
\end{tabularx}
\end{table}




\begin{table}[ht]
\centering
\small\renewcommand{\arraystretch}{1.4}  
\rowcolors{1}{tablerowcolor}{tablebodycolor}
%
\captionabove[PageRank for User-Repository stars]{Top 20 central GitHub repositories by PageRank.}
\label{follow_pagerank_table}
%
\begin{tabularx}{0.8\textwidth}{lXXX}
\hline
\rowcolor{tableheadcolor}
Repository & PageRank \\
\hline
\url{https://github.com/vhf/free-programming-books} & 455.07 \\
\url{https://github.com/twbs/bootstrap} & 335.46 \\
\url{https://github.com/jquery/jquery} & 289.42 \\
\url{https://github.com/resume/resume.github.com} & 251.99 \\
\url{https://github.com/mandatoryprogrammer/Octodog} & 233.41 \\
\url{https://github.com/angular/angular.js} & 202.18 \\
\url{https://github.com/mbostock/d3} &149.55 \\
\url{https://github.com/torvalds/linux} & 133.48 \\
\url{https://github.com/FortAwesome/Font-Awesome} & 121.47 \\
\url{https://github.com/twitter/bootstrap} & 111.42 \\
\url{https://github.com/laravel/laravel} & 106.75 \\
\url{https://github.com/papers-we-love/papers-we-love} & 102.25 \\
\url{https://github.com/joyent/node} &101.27 \\
\url{https://github.com/rethinkdb/rethinkdb} & 92.34 \\
\url{https://github.com/neovim/neovim} & 91.52 \\
\url{https://github.com/libgit2/libgit2} & 90.99 \\
\url{https://github.com/rogerwang/node-webkit} & 88.23 \\
\url{https://github.com/github/gitignore} & 88.08 \\
\url{https://github.com/dypsilon/frontend-dev-bookmarks} & 86.73 \\
\url{https://github.com/zurb/foundation} & 84.25 \\
\hline
\end{tabularx}
\end{table}


\section{Evaluation Metrics}

Each method described above in section (PROPER SECTION CITATION) is evaluated using the same evaluation metrics for comparison.

\subsection{Precision / Recall}

\subsection{Hit Ratio}

\subsection{Operator Receiver Curve}

