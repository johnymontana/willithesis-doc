\IfDefined{colorlet}{
   \colorlet{colorlstNumber}{white!50!black!100}
}
\IfPackageLoaded{listings}{%

\lstdefinestyle{lstStyleBase}{
%%% appearance
   ,basicstyle=\small\ttfamily % Standardschrift
%%%  Space and placement
   ,floatplacement=tbp    % is used as float place specifier
   ,aboveskip=\medskipamount % define the space above and 
   ,belowskip=\medskipamount % below displayed listings.
   ,lineskip=0pt          % specifies additional space between lines in listings.
   ,boxpos=c              % c,b,t
%%% The printed range
   ,showlines=false       % prints empty lines at the end of listings
%%% characters
   ,extendedchars=true   % allows or prohibits extended characters 
                         % in listings, that means (national)
                         % characters of codes 128-255. 
   ,upquote=true         % determines printing of quotes
   ,tabsize=2,           % chars of tab
   ,showtabs=false       % do not show tabs
   ,showspaces=false     % do not show spaces
   ,showstringspaces=false % do not show blank spaces in string
%%% Line numbers
   ,numbers=left         % left, right, none
   ,stepnumber=1         % seperation between numbers
   ,numberfirstline=false % number first line always
   ,numberstyle=\tiny\color{colorlstNumber}    % style of numbers
   ,numbersep=5pt        % distance to text
   ,numberblanklines=true %
%%% Captions
   ,numberbychapter=true %
   ,captionpos=b         % t,b
   ,abovecaptionskip=\smallskipamount % the vertical space respectively above 
   ,belowcaptionskip=\smallskipamount % or below each caption
%%% Margins and line shape
   ,linewidth=\linewidth % defines the base line width for listings.  
   ,xleftmargin=0pt      % extra margins
   ,xrightmargin=0pt     %
   ,resetmargins=false   % indention from list environments like enumerate 
                         % or itemize is reset, i.e. not used.
   ,breaklines=true      % line breaking of long lines.
   ,breakatwhitespace=false % allows line breaks only at white space.
   ,breakindent=0pt     % is the indention of the second, third, ...  
                         % line of broken lines.
   ,breakautoindent=true % apply intendation
   ,columns=flexible     %
   ,keepspaces=true      %
}

\lstset{style=lstStyleBase}

\lstdefinestyle{lstStyleFramed}{%
%%% Frames
   ,frame=single         % none, leftline, topline, bottomline, lines
                         % single, shadowbox
   ,framesep=3pt 
   ,rulesep=2pt          % control the space between frame and listing 
                         % and between double rules.
   ,framerule=0.4pt      % controls the width of the rules.
}

% do not activate!:
% frames in fancyvrb are printed out wrong!
% \IfPackageLoaded{fancyvrb}{\lstset{fancyvrb=true}}

% correct utf8 umlaute
\lstset{literate=%
  {Ö}{{\"O}}1
  {Ä}{{\"A}}1
  {Ü}{{\"U}}1
  {ß}{{\ss}}2
  {ü}{{\"u}}1
  {ä}{{\"a}}1
  {ö}{{\"o}}1
  {€}{{\geneuro{}}}1
}

%% provide command \addmoretexcs
% Code from Heiko Oberdiek, see
% http://tex.stackexchange.com/questions/84207/define-moretexcs-listings
% 
% Description:
% The following example defines macro \addmoretexcs. The optional argument
%  specifies the dialect (default is common). The language is loaded if it is
%  not yet available. Then the language definition, internally stored in
%  \lstlang@<language>$<dialect>, is extended by setting the additional
%  moretexcs list
% ------------>
\makeatletter
\newcommand*{\addmoretexcs}[2][common]{%
  \lowercase{\@ifundefined{lstlang@tex$#1}}{%
    \lstloadlanguages{[#1]TeX}%
  }{}%
  \lowercase{\expandafter\g@addto@macro\csname lstlang@tex$#1\endcsname}{%
    \lstset{moretexcs={#2}}%
  }%
}
\makeatother
% <------------

\IfDefined{colorlet}{
  % style files make use of colors and require \colorlet
  \colorlet{lstcolorStringLatex}{green!40!black!100}
\colorlet{lstcolorCommentLatex}{green!50!black!100}
\definecolor{lstcolorKeywordLatex}{rgb}{0,0.47,0.80}

% define useless command for checking the
% existens of this style
\newcommand{\lstStyleLaTeX}{\relax}
% define style
\lstdefinestyle{lstStyleLaTeX}{%
   ,style=lstStyleBase
%%% colors
   ,stringstyle=\color{lstcolorStringLatex}%
   ,keywordstyle=\color{lstcolorKeywordLatex}%
   ,commentstyle=\color{lstcolorCommentLatex}%
   ,% backgroundcolor=\color{codebackcolor}%
%%% Frames
   ,frame=single%
   ,%frameround=tttt%
   ,%framesep = 10pt%
   ,%framerule = 0pt%
   ,rulecolor = \color{black}%
%%% language
   ,language = [LaTeX]TeX%
%%% commands
% moved to: listings-latex-texcs.tex
}

\ifcsdef{addmoretexcs}{%
% LaTeX programming
\addmoretexcs[LaTeX]{setlength,usepackage,newcommand,renewcommand,providecommand,RequirePackage,SelectInputMappings,ifpdftex,ifpdfoutput,AtBeginEnvironment,ProvidesPackage}
% other commands
\addmoretexcs[LaTeX]{maketitle,text,includegraphics,chapter,section,subsection,
subsubsection,paragraph,textmu,enquote,blockquote,ding,mathds,ifcsdef,Bra,Ket,Braket,subcaption,lettrine,mdfsetup,captionof,listoffigures,listoftables,tableofcontents,appendix,url}
% tables
\addmoretexcs[LaTeX]{newcolumntype,rowfont,taburowcolors,rowcolor,rowcolors,bottomrule,toprule,midrule}
% hyperref
\addmoretexcs[LaTeX]{hypersetup}
% glossaries
\addmoretexcs[LaTeX]{gls,printglossary,glsadd,newglossaryentry,newacronym}
% Koma
\addmoretexcs[LaTeX]{mainmatter,frontmatter,geometry,KOMAoptions,setkomafont,addtokomafont}
% SI, unit
\addmoretexcs[LaTeX]{si,SI,sisetup,unit,unitfrac,micro}
% biblatex package
\addmoretexcs[LaTeX]{newblock,ExecuteBibliographyOptions,addbibresource}
% math packages
\addmoretexcs[LaTeX]{operatorname,frac,sfrac,dfrac,DeclareMathOperator,mathcal,underset}
% demo package
\addmoretexcs[LaTeX]{democodefile,package,cs,command,env,DemoError,PrintDemo}  
% tablestyles
\addmoretexcs[LaTeX]{theadstart,tbody,tsubheadstart,tsubhead,tend}
% code section package
\addmoretexcs[LaTeX]{DefineCodeSection,SetCodeSection,BeginCodeSection,EndCodeSection}
% template tools package
\addmoretexcs[LaTeX]{IfDefined,IfUndefined,IfElseDefined,IfElseUndefined,IfMultDefined,IfNotDraft,IfNotDraftElse,IfDraft,IfPackageLoaded,IfElsePackageLoaded,IfPackageNotLoaded,IfPackagesLoaded,IfPackagesNotLoaded,ExecuteAfterPackage,ExecuteBeforePackage,IfTikzLibraryLoaded,IfColumntypeDefined,IfColumntypesDefined,IfColorDefined,IfColorsDefined,IfMathVersionDefined,SetTemplateDefinition,UseDefinition,IfFileExists,iflanguage}
% tablestyles
\addmoretexcs[LaTeX]{setuptablefontsize,tablefontsize,setuptablestyle,tablestyle,setuptablecolor,tablecolor,disablealternatecolors,tablealtcolored,tbegin,tbody,tend,thead,theadstart,tsubheadstart,tsubhead,theadrow,tsubheadrow,resettablestyle,theadend,tsubheadend,tableitemize,PreserveBackslash}
% todonotes
\addmoretexcs[LaTeX]{todo,missingfigure}
% listings
\addmoretexcs[LaTeX]{lstloadlanguages,lstdefinestyle,lstset}
% index
\addmoretexcs[LaTeX]{indexsetup}
% glossaries
\addmoretexcs[LaTeX]{newglossarystyle,glossarystyle,deftranslation,newglossary}
% tikz
\addmoretexcs[LaTeX]{usetikzlibrary}
% color
\addmoretexcs[LaTeX]{definecolor,colorlet}
% caption
\addmoretexcs[LaTeX]{captionsetup,DeclareCaptionStyle}
% floatrow
\addmoretexcs[LaTeX]{floatsetup}
% doc.sty
\addmoretexcs[LaTeX]{EnableCrossrefs,DisableCrossrefs,PageIndex,CodelineIndex,CodelineNumbered}
% refereces
\addmoretexcs[LaTeX]{cref,Cref,vref,eqnref,figref,tabref,secref,chapref}
%
}{} % end of \ifcsdef

\lstloadlanguages{[LaTeX]TeX}
  %\colorlet{colorlstStringCpp}{green!40!black!100}
\colorlet{colorlstCommentCpp}{green!50!black!100}
\colorlet{colorlstBackgroundCpp}{white!100}
\definecolor{colorlstStringCpp}{rgb}{0,0.47,0.80}

%% \colorlet{colorlstStringCpp}{green!100!black!100}
%% \colorlet{commencolor}{green!100!red!50!black!100}
%\definecolor{commencolor}{rgb}{0.0,0.5,0.0}
\definecolor{colorlstKeywordCpp}{rgb}{0.4,0.4,0.0}

% define useless command for checking the
% existens of this style
\newcommand{\lstStyleCpp}{\relax}
% define style
 \lstdefinestyle{lstStyleCpp}{%
   ,style=lstStyleBase
%%% Numbers
   ,,stepnumber=1%
%%% colors
   ,keywordstyle=\textbf\ttfamily\color{colorlstKeywordCpp}%
   ,identifierstyle=\ttfamily%
   ,commentstyle=\color{colorlstCommentCpp}%
   ,stringstyle=\ttfamily\color{colorlstStringCpp} %\color[rgb]{0,0.5,0},
   ,backgroundcolor=\color{colorlstBackgroundCpp}%
%%% Frames
   ,frame=single%
   ,%frameround=tttt
   ,%framesep = 10pt
   ,%framerule = 0pt
%%% language
   ,language = C++%
   ,otherkeywords={string},
%%% Comments
   ,morecomment=[l][\color{colorlstCommentCpp}]{//},%
   ,morecomment=[s][\color{colorlstCommentCpp}]{/*}{*/}%
}
 
\lstloadlanguages{
   ,C++
   ,[Visual]C++
   ,[ISO]C++
  % Javascript
}
}

% load language used in document. 
% (LaTex and C++ already loaded)
\lstloadlanguages{
   %,[LaTeX]TeX
   %,C++
   %,[Visual]C++
   %,[ISO]C++   
   %,[Visual]Basic
   %,Pascal
   %,C
   %,XML
   %,HTML
}

}% End of \IfPackageLoaded